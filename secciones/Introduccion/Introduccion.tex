\section{Introducción}
Todos los días los negocios y empresas trabajan por varias horas en distintas tareas y proyectos, cada uno enfocado a diferentes áreas y beneficios, si bien tienen su métodos o técnicas para llevar a cabo sus tareas, la necesidad de agilizar esos procesos para que se realicen en menos tiempo y costos aumenta cada vez más y es aquí donde entra el software, las computadoras son aparatos inteligentes que pueden realizar cálculos grandes en cuestión de segundos, ahora si se diseña un sistema correcto y de calidad se puede mejorar la atención al cliente de un negocio, control sobre inventario e incluso la comunicación de datos. Nuestra propuesta es mejorar el control de una lavandería mediante un sitio web desarrollado en un Framework para aprovechar la tecnología con la que se cuenta y satisfacer las necesidades de nuestro cliente. 




\subsection{Problematica}
Una lavandería en el Fracc. Colinas de Plata en Mineral de la Reforma en la que sus clientes asisten a su local y dejan sus prendas para que se lleve a cabo un lavado, se le entrega un comprobante con su numero de pedido y fecha de entrega, pero muchos de sus clientes olvidan recoger sus pedidos o incluso olvidan ropa en el local, por lo que el cliente requiere de un sistema que notifique a sus clientes cuando sus pedidos estén listos.
\subsubsection{Descripción del cliente}
Un negocio de lavandería ubicado en el Fracc. Colinas de Plata en Mineral de la Reforma a cargo de la señorita Casandra Rosas Juárez y en tiempo parcial cuenta con una empleada. Ofrece servicios de lavado y secado únicamente para los distintos tipos de ropa como son algodón, poliéster y seda.

\subsubsection{Descripción de la propuesta o solución}
La oferta que se le hizo a nuestro cliente es un sitio web que cuente con un menú con secciones como son:

\begin{itemize}
\item \textbf{Pedidos pendientes}: El cliente podrá visualizar información relacionada con sus pedidos que aún no son entregados o están en proceso de lavado.

\item \textbf{Pedidos Entregados}: Se mostrarán los pedidos que han sido finalizados y entregados con anterioridad al cliente.

\item \textbf{Nuevo pedido}: Formulario donde se solicitará el numero de prendas, tipo de prendas, fecha y hora de entrega para que posteriormente sea enviado como petición de un nuevo pedido.
\end{itemize}



Además de que a los clientes se les notificará por correo electrónico cuando su(s) pedidos estén listos, de esta manera tanto los clientes como el administrador del negocio puede llevar un mejor control de sus pedidos.


\subsubsection{Herramientas y métodos propuestos}


\textbf{Framework Laravel}: Es un framework de código abierto para desarrollar aplicaciones y servicios web con PHP 5 y PHP 7. Tiene como objetivo ser un framework que permita el uso de una sintaxis elegante y expresiva para crear código de forma sencilla y permitiendo multitud de funcionalidades. Intenta aprovechar lo mejor de otros frameworks y aprovechar las características de las últimas versiones de PHP.

\textbf{WorkBench (MYSQL)}: Es una herramienta visual de diseño de bases de datos que integra desarrollo de software, Administración de bases de datos, diseño de bases de datos, gestión y mantenimiento para el sistema de base de datos MySQL

\textbf{XAMPP}: Es un paquete de software libre, que consiste principalmente en el sistema de gestión de bases de datos MySQL, el servidor web Apache y los intérpretes para lenguajes de script PHP y Perl.

\textbf{Sublime-Text}:  Es un editor de texto y editor de código fuente está escrito en C++ y Python para los plugins. Se puede descargar y evaluar de forma gratuita. Sin embargo no es software libre o de código abierto y se debe obtener una licencia para su uso continuado, aunque la versión de evaluación es plenamente funcional y no tiene fecha de caducidad.

\textbf{LateX}: Es un sistema de composición de textos, orientado a la creación de documentos escritos que presenten una alta calidad tipográfica. Por sus características y posibilidades, es usado de forma especialmente intensa en la generación de artículos y libros científicos que incluyen, entre otros elementos, expresiones matemáticas.

\textbf{Bootstrap-3.3.7}: Herramienta muy utilizada para la creación de páginas web, y aunque mucha gente se va por la herramienta por la parte de los botones, los label y entre otras cosas, la verdad es que bootstrap es más que eso ya que esta herramienta en realidad lo mejor que te brinda es la rejilla adaptable capaz de funcionar en dispositivos móviles como tabletas o teléfonos. Ocupa lenguajes comunes los cuales son HTML, PHP, CSS y JS y esta herramienta tiene tantos componentes que tiene un modo de descarga en el cual puedes seleccionar solo los componentes que vas a utilizar para que los archivos

\textbf{Microsoft Word}:Potente herramienta desarrollada por Microsoft integrada en el paquete de Office que nos permite crear documentos, libros, artículos etc.
