\section{Desarrollo del sistema}
\subsection{Descripción de las herramientas ocupadas.}
•	LARAVEL: Es un framework de código abierto para desarrollar aplicaciones y servicios web con PHP 5 y PHP 7. Tiene como objetivo ser un framework que permita el uso de una sintaxis elegante y expresiva para crear código de forma sencilla y permitiendo multitud de funcionalidades.
•	MySQL: es un sistema de gestión de base de datos relacional (RDBMS) de código abierto, basado en lenguaje de consulta estructurado (SQL). MySQL se ejecuta en prácticamente todas las plataformas, incluyendo Linux, UNIX y Windows.
•	PHP: es un lenguaje de código abierto muy popular especialmente adecuado para el desarrollo web y que puede ser incrustado en HTML.
•	Arquitectura Modelo Vista Controlador: es un estilo de arquitectura de software que separa los datos de una aplicación, la interfaz de usuario, y la lógica de control en tres componentes distintos. Se trata de un modelo muy maduro y que ha demostrado su validez a lo largo de los años en todo tipo de aplicaciones, y sobre multitud de lenguajes y plataformas de desarrollo.


\subsection{Programación}